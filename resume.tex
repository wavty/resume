% !TEX TS-program = xelatex
% !TEX encoding = UTF-8 Unicode
% !Mode:: "TeX:UTF-8"

\documentclass{resume}
\usepackage{zh_CN-Adobefonts_external} % Simplified Chinese Support using external fonts (./fonts/zh_CN-Adobe/)
% \usepackage{zh_CN-Adobefonts_internal} % Simplified Chinese Support using system fonts
\usepackage{linespacing_fix} % disable extra space before next section
\usepackage{cite}

\begin{document}
\pagenumbering{gobble} % suppress displaying page number

\name{陶波}

\basicInfo{
  \email{mtaobo@yeah.net} \textperiodcentered\ 
  \phone{151-0560-3818} \textperiodcentered\ 
  \github[wavty]{https://github.com/wavty/} \textperiodcentered\ 
  \homepage[ombak.cn]{https://www.ombak.cn/} 
}
 
\section{\faGraduationCap\  教育背景}
\datedsubsection{\textbf{合肥工业大学}, 安徽, 合肥}{2017 -- 2021, 本科, 交通运输}
\bigbreak

\section{\faHeartO\ 工作经历}
\datedline{\textit{腾讯},后台开发工程师}{2021年8月 -- 2023年1月}
\datedline{可观测平台、代码仓库、代码扫描模块后台开发}{深圳}

\section{\faCogs\ IT 技能}
% increase linespacing [parsep=0.5ex]
\begin{itemize}[parsep=0.5ex]
  \item 编程语言: Go、Python、C++ 
  \item 开发平台: Linux、k8s
  \item 中间件: 消息队列、MySQL、Redis
  % \item 开发: 代码托管和扫描、可观测
\end{itemize}

\section{\faUsers\ 项目经历}
\datedsubsection{\textbf{可观测业务后台开发}, 深圳}{2021年9月 -- 2022年1月}
\role{技术栈}{Golang, Prometheus, Grafana, 微服务}
\begin{onehalfspacing}
职责: 基于腾讯云可观测生态工具,参与建设腾讯云扣钉可观测平台。

背景: 腾讯云扣钉可观测平台基于云监控、云日志、云APM搭建,支持业务快速接入,降低用户接入相关产品的门槛。

任务: 
\begin{itemize}
  \item 监控方面使用 Prometheus + Grafana 做 Metrics 的存储、展示;
  \item 使用云日志 CLS 进行日志数据的存储和分析,平台基于 sidecar 进行日志数据的上报,不对业务产生侵入;
  \item 使用云APM对符合 Opentelemetry 的规范的Trace数据进行收集和分析,同时可结合日志和指标监控快速定位故障;
\end{itemize}

结果: 至2021年末已有腾讯会议、教育等多个业务部门接入平台。
\end{onehalfspacing}

\datedsubsection{\textbf{腾讯云扣钉代码仓库架构升级}}{2022年2月 -- 2022年11月}
\role{技术栈}{Golang, Linux, 微服务}
\begin{onehalfspacing}
职责: 参与扣钉代码仓库后台服务的开发工作,确保服务稳定、高效运行。

背景: 原代码仓库数据分片之后,一个集群只有一台主机提供读写服务,同时使用冷备的方式进行备份,当访问量骤增之后会出现服务不稳定的情况。

目标: 对代码仓库底层数据进行分片,并做到读写分离。

任务: 
\begin{itemize}
  \item 由主机写入数据,并通过hooks向从机发起同步任务,同步任务如果失败则执行预设的补偿机制;
  \item 从机收到同步请求之后执行同步操作;
  \item 当收到新的读请求的时候,会对主机和从机进行一致性校验,如果校验通过会从从机读取数据。
\end{itemize}

结果: 在后期的压测过程中,发现同步请求中位数为65毫秒,平均备机可读率为96\%。
\end{onehalfspacing}

\datedsubsection{\textbf{腾讯云扣钉代码扫描支持节点模式}}{2022年10月 -- 至今}
\role{技术栈}{Python, Linux, 微服务}
\begin{onehalfspacing}

背景: 原代码扫描仅可以通过持续集成运行,且与ci模块强耦合。

目标: 通过对扫描模块进行改造,使其自行维护节点管理能力,与ci解耦。

任务: 
\begin{itemize}
  \item 支持通过 helm 的方式接入新的节点;
  \item 支持前端对节点进行编辑以及设置扫描任务的执行环境标签;
  \item 重构扫描任务的执行逻辑,扫描任务分发到对应的节点执行。
\end{itemize}

结果: 通过启动分布式分析节点集群,不仅做到了与其他模块的解耦,同时也支持专机专用,在一定程度上提高了扫描效率。
\end{onehalfspacing}

\end{document}
